\chapter{绪论}
\section{引言}
文字和语音一直都是自然语言处理领域的研究对象。无论是比较早期的句法、词法分析,音素和发声研究,还是诞生稍晚的针对语音内容的识别、拼接合成等,对于文字和语音内容的研究可称得上互为表里。语音是文字可感知的信号载体、是人类使用最频繁、最自然的``通信接口''之一,文字则是语音内容的本质表示。

机器翻译一直以来都是自然语言处理领域针对不同语言文字的一项重要技术。对于机器翻译的研究最早始于20世纪30年代,它的发展一直与计算机技术、语言学和信息论等学科密切相关。从早期的字典匹配、结合专家知识的规则翻译等朴素方法,到结合概率统计学和语言学的统计机器翻译,再到近年来随着运算硬件性能提高和深度神经网络发展而兴起的神经机器翻译,机器翻译技术取得了世人瞩目的成功,这项技术本身也逐渐从学界中的理论发展走向工业界的落地实践。
自动歌曲翻译是神经机器翻译在这一基础上针对非常规语体的拓展性研究,这一任务旨在快速地、自动地、高质量地将歌曲的歌词文本翻译到另外一种语言中,同时要求翻译后歌词文本搭配相应旋律依然能以歌曲的形式呈现出来,即翻译后歌词仍能以某种方式来进行艺术性的演唱并传达原意。除了歌词近似诗词的语体问题以外,这样的翻译也相比于一般的机器翻译任务增加了一项需要考虑的旋律乐谱信息。

语音合成任务旨在将文字合成为可理解的自然语音,是一个比翻译更加多元的任务。
语音合成研究需要自然语言处理和人类发声的知识,涉及多个学科,包括语言学、声学、数字信号处理等。
歌声合成(Singing Voice Sythesis,SVS)任务则是由语音合成任务衍生而来。语音合成是仅以文本为输入、以梅尔频率特征图谱或直接以声波波形为输出的生成任务,歌声合成和语音合成任务不同的是,其文本发声时所对应的音高和时长都为设定好的乐谱所限制。
歌声合成使得直观地呈现歌曲翻译结果成为可能。同时,由于歌声合成可以承接歌曲翻译的输出而连接成为完整的级联式歌曲到歌声的翻译系统,这为歌曲翻译的实际应用打下了坚实的基础。自动歌曲翻译搭配良好的歌声合成效果进行翻唱自动合成,那么听众无需学习多国语言、乐理知识或具备演唱能力就能欣赏来自不同语言不同文化的经典歌曲的母语翻唱版本。
\section{国内外研究现状}
\subsection{自动歌曲翻译国内外研究现状}
歌曲翻译作为人工翻译子任务的研究历史很长,语言学领域对此任务也有体系化的研究。但当自然语言和翻译研究进入深度学习时代时,自动化歌曲翻译研究却相对发展缓慢,目前仅有极少量工作\citep{gagast}对这一方向进行过探讨。此外,翻译领域如WMT14、WMT16等多语言翻译公开数据集层出不穷,但涉及词曲类的公开翻译语料则非常寥寥。歌曲本身的电子化需要有一定专业知识的人员完成,即使借助于神经网络模型,在实际错误率较高的情况下也需要人员校对,因此研究本身就具有一定的门槛。在充分感受、理解歌曲的基础上进行的翻译则更有难度。
\subsection{歌声合成国内外研究现状}
近年来,随着深度学习研究和硬件发展,歌声合成技术因其产生自然而富有表现力的歌声的潜力而越来越受到研究界和娱乐行业的关注。歌声合成技术使用的模型的一种基本框架是一种由两种模型构成的两阶段方法~\citep{nakamura2019singing,lee2019adversarially,blaauw2020sequence,ren2020deepsinger,chen2020hifisinger}。两个模型通常由一个声学模型和一个声码器组成,这个声学模型用于根据歌谱包含的歌词、音符音调和音符时值信息生成声学特征(例如梅尔频谱)供声码器使用,该声码器用于将声学模型产生的声学特征转换为波形。

在以往研究中,声学模型这一部分主要利用一些简单的损失函数(例如L1,即一范数;或L2,即二范数)来重构声学特征。然而,基于简单损失函数的优化基于不正确的单峰分布假设,所以会导致输出的频谱有模糊和过度平滑的问题。尽管现有方法试图通过生成对抗训练(GAN)~\citep{lee2019adversarially,chen2020hifisinger}来解决此问题,但由于对抗训练的交替更新和鉴别器训练稳定性问题,这样的对抗训练经常失败,很不稳定,需要较多人力参与调整。这些问题阻碍了歌声合成的发展。

最近,有一类高度灵活且易于处理的生成模型研究崭露头角,即扩散概率模型(也称扩散模型)~\citep{sohl2015deep,Ho2020ddpm,song2021denoising}。扩散模型由两个过程组成:扩散过程和反向过程(也称为去噪过程),扩散过程是一个具有固定参数的马尔可夫链,它通过逐渐添加高斯噪声将复杂数据转换为高斯分布;而反向过程是由神经网络实现的马尔可夫链,用于学习从高斯白噪声中迭代地恢复为原始数据的去噪过程。通过隐式地优化数据似然的变分下界(Evidence Lower Bound,ELBO),扩散模型可以稳定地进行训练。现有最新工作证明,扩散模型可以在图像生成~\citep{Ho2020ddpm,song2021denoising}和神经声码器~
\citep{chen2021wavegrad,kong2021diffwave}这样的生成任务中取得很有潜力的实验结果。

基于以上考虑,本文以歌声合成音频质量提升和实际应用为目标,研究基于扩散模型构建声学模型的技术方案,以克服之前工作提出的做法的局限和问题。
\section{本文工作的研究内容及意义}
歌曲翻译技术是人类为了攀登更高层次的跨文化交流之巴别塔而做出的很有意义的技术努力。
\begin{figure}[htbp]
  \includegraphics[width=0.99\textwidth]{figure/ast/exp.pdf}
  \caption{以\textit{Rolling In the Deep}一曲中``But you play it to the beat''一句的完整歌曲翻译为例。}
  \label{fig:task_exp}
\end{figure}
然而,尽管机器翻译(Machine Translation,MT)技术,尤其是神经机器翻译~\citep{nmt,vaswani2017attention,hassan2018achieving}(Neural Machine Translation,NMT)自诞生以来已经取得了长足进步,也吸引了许多领域研究者投身其中,自动歌曲翻译在自然语言处理学界中却并未得到充分的研究探索。这其中客观存在的一些挑战包括缺乏收集平行歌词和对齐数据的高效方式、难以对文本和旋律之间的复杂交互进行建模以及没有对乐谱规定的演唱方式进行直观评估的方式。歌曲翻译虽然与文本翻译密切相关,但本质上是一项更复杂的任务。除了在翻译中如用词和词序这样的语体问题带来的额外翻译考虑之外,歌曲的人工翻译者还需要具有目标语言的背景,能理解源语言并作出目标语言中诗意化的表达。此外,如图\ref{fig:task_exp}所示,翻译的歌词需要与旋律合理地对齐来保持歌曲的美感,这是歌曲翻译中不可缺少的要素~\citep{three_d_of_singability}。高质量的歌声合成效果应能细致地预测出谐波间的细节、准确地建模人声的高频部分以达到体现人声特点、突出高音和共鸣等艺术感的歌声效果。

此前,如前节所述,学界也探索过歌声合成这一技术来自动化地合成歌曲的人声演唱,并提出了一些在给定歌词和歌谱的情况下产生具有真实人声音色的、自然的、准确的歌声的方法。这样的方法不但使得对歌曲翻译结果方便而直观的评估成为可能,而且也为自动写歌谱曲、自动歌曲翻译这样的研究任务的实际落地奠定了基础。然而,自动歌曲翻译方向上的研究和歌声合成相比很少,也未有工作探索过为歌曲翻译结果进行的翻唱歌声合成。作为目前为数不多的工作之一,\citet{gagast}专注于通过在神经机器翻译的推理过程中施加特定约束来匹配有声调语言的翻译目标词语和旋律的音调、节奏等来得到更加合适、不易造成误解的翻译歌词。然而,这篇工作直接使用文本翻译模型并对音符和字符之间对齐的严格规定一对一的匹配,无法捕捉到歌曲翻译更复杂的本质——即歌词和歌词-旋律对齐之间的关系。虽然音符的数量可以当作是翻译长度的一个简单上限,但正如\citet{interplay_lyrics_melody}一文中所观察到的现象,歌词和旋律之间的微妙对齐不应仅为简单而严格的规则所决定。针对翻唱的歌声合成也需兼容翻译结果,并提升合成质量以求达到与原端相匹配的翻唱合成效果。

为了解决上述技术挑战,本文提出了带有自适应分组的歌词-旋律共同翻译模型,这是自动歌词翻译问题的第一个完整的技术解决方案,通过在基于Transformer的编码器-解码器框架内对歌词翻译和歌词-旋律对齐进行联合建模,本文提出的模型翻译出的歌曲既忠实于原歌词,又符合旋律,无论是客观指标还是主观评测都显示出模型翻译表现的优越性。为了承接共同翻译模型的翻唱结果输出,自动地、快速地合成良好的翻唱歌声,本文提出了基于扩散模型和对抗训练的歌声合成声学模型以求高质量地自动合成歌曲翻译的翻唱结果。
\section{章节安排}
本文分为5个章节展开,各章节组织如下:

第1章:绪论。第一章节介绍了本文主要研究内容,即自动歌曲翻译技术和歌声合成技术的定义、应用和发展情况、自动歌曲翻译和歌声合成的国内外研究现状、以及自动歌曲翻译和歌声合成的研究背景及意义。另外,第一章节还阐述了本文将基于Transformer的编码器-解码器模型和歌词和歌词-旋律对齐的关系来研究自动歌曲翻译任务、基于扩散模型来研究歌声合成任务,并探究本章提出的音乐、轻量级对齐模块对于自动歌曲翻译和扩散模型、浅扩散机制对于歌声合成效果的影响。

第2章:相关研究介绍。第二章节分两部分分别介绍了本文所研究内容的相关技术综述,包括自动歌曲翻译技术所涉及的歌词生成、限制性翻译、歌词对齐预测和歌声合成技术涉及的声学模型以及扩散模型。

第3章:自动歌曲翻译研究。第三章节主要介绍本文提出的一种能很好适配基于Transformer的编码器-解码器的自回归神经机器翻译框架的歌词和歌词-旋律对齐共同翻译模型,介绍本文提出的针对歌词翻译任务的音符嵌入表示模块和对齐解码器,以及详细叙述模型结构和设计,最后介绍本文的数据来源和数据集构建方式、数据预处理方法、实验中的超参数,并展示本文提出的翻译模型其它类似模型的对比实验结果和模块消融实验结果。

第4章:翻唱歌声自动合成研究。第四章节主要介绍本文提出的基于扩散模型和对抗训练构建的声学模型,以及详细叙述模型结构和设计,最后介绍本文的数据来源和数据集构建方式、数据预处理方法、实验中的超参数,并展示本文提出的声学模型和其它声学模型的对比实验结果和模块消融实验结果。

% 第五章:歌曲到歌声翻译系统实践。
第5章:总结和展望。第五章总结了本文的主要研究内容,指出本文进行的研究的不足之处,讨论在本文探讨的自动歌曲翻译技术和翻唱自动合成技术工作基础上可进一步研究的方向。
\section{章节小结}
本章主要介绍了自动歌曲翻译技术和歌声合成技术的定义、意义和重要性,并简要地阐述了国内外现有方法的不足之处和本文工作的内容和意义。

\chapter{相关研究介绍}
本文的研究对象涉及自然语言处理的多个子领域,本章将分领域分别介绍歌曲歌词生成及限制性翻译和歌声合成相关技术的研究现状和近期进展。
歌曲歌词生成及限制性翻译目前有多个技术路线,分别着重于对自回归翻译的不同阶段施加限制,另外也有一些条件性歌词生成和对齐的相关技术与自动歌曲翻译这一任务相关。
歌声合成技术自语音合成发展而来,在语音合成方面,许多工作尝试以各类合成模型为基础搭建声学模型。这些模型因各自原理和结构不同,在模型规模、推理速度、训练难易度和合成表现上也各有千秋。
以下分别对这些子领域进行阐述。
\section{歌曲歌词生成及限制性翻译研究介绍}
歌词自动翻译研究历经基于规则的方法的阶段、统计机器翻译方法阶段和使用具有节奏和词汇句法约束的有限状态机阶段\citep{spanish_verse, Manurung2004AnEA, He_Zhou_Jiang_2012},近年来也逐渐开始引入神经网络模型向神经机器翻译靠拢
\citep{ghazvininejad-etal-2016-generating,ghazvininejad-etal-2017-hafez, ghazvininejad-etal-2018-neural}。
在语言学研究中,传统人工歌曲翻译研究通过利用语言学知识在歌词翻译和歌词旋律对齐方面都取得了一些进展
~\citep{interplay_lyrics_melody,low_2003,low2008translating,low_2022,three_d_of_singability,trans_of_music}.
当然,这些研究针对的对象都是专业歌手创作的歌曲,这些方法追求在一些有代表性的曲目中进行高质量的歌词翻译和歌词旋律对齐,力求达到``信、达、雅''的三重境界。
\begin{figure}[htbp]
  \centering
  \includegraphics[width=0.99\textwidth]{figure/related/teacher-forcing.pdf}
  \caption{Teacher-forcing训练方式图示}
  \label{fig:tf_train}
\end{figure}
\subsection{基于序列生成的歌词生成、限制性生成和翻译}
\subsubsection{限制性生成和翻译研究}
在近来基于神经机器翻译的工作\citep{gagast}中,自动歌曲翻译大多被当作一种存在一定限制条件的文本翻译任务进行建模。目前,文本翻译作为一种序列生成任务,大都采用自回归式的解码方式,而以Teacher-forcing方式进行训练以加快收敛。
基于此,很多工作~\citep{hokamp-liu-2017-lexically,lakew-etal-2019-controlling,li-etal-2020-rigid,zou_controllable}尝试仅在解码过程中直接对解码搜索时的评分施加目的性限制来进行重评分,让符合限制的结果得分更高,不符合的得分减少。如限制符合诗词格式、限制符合某些语法规则等。
解码时对结果的评分原本只有来自神经网络训练出的语言模型根据编码器的输入和已经解码出的前文对当前位置应解码结果的概率估计:
\begin{equation}
  P(y_t|y_0,y_1......y_{t-1}, X)
\end{equation}
现在则需要根据是否符合限制的判断进行重评分:
\begin{equation}
  P(y_t|y_0,y_1......y_{t-1}, X)+f(y_0,y_1......y_{t-1}, \mbox{restrictions})
\end{equation}
xxx中使用
\begin{figure}[htbp]
  \includegraphics[width=0.99\textwidth]{figure/related/decoded_constrain.pdf}
  \caption{title}
\end{figure}
此方面工作的探索证明了这类比较直接的做法大部分都比较有效,而且对于一些本质比较简单的限制来说,这种做法所需的编码工作量小,实施起来非常方便,而且无需对已经训练好的神经网络模型做其他调整。

除此之外,很多工作尝试在训练过程中施加约束,如在输入中添加与格式限制有关的嵌入表示进行监督以在推理时控制解码~\citep{li-etal-2020-rigid}、引入特殊词语以达到长度控制的目的~\citep{lakew-etal-2019-controlling,saboo-baumann-2019-integration}等。这些方法通过在输入时引入额外的条件作为控制信息,在训练时通过相应的结果监督来使得模型依赖控制信息对最终结果产生的影响,进而在推理时通过提供不同的控制信息来控制解码结果。这些数据驱动的方法同样表现出良好的性能,且能施加更加复杂的限制,模型表现更可靠。但是和仅干预解码过程的做法相比,限制的变动都需要设计和训练模型学习限制的方式。
% \begin{figure}[htbp]
%   \includegraphics[width=0.99\textwidth]{figure/related/train_constrain.pdf}
%   \caption{title}
% \end{figure}
\subsubsection{自动歌词生成研究}


在下一章中,本文将提出歌词旋律对齐和歌词共同翻译模型,在翻译的语料上进行翻译域偏移适应,同时也对翻译文本结果进行长度限制。
\subsection{歌词-旋律对齐预测方法}
\subsubsection{基于注意力机制的歌词-旋律对齐预测}
包含歌词旋律对齐预测的歌词生成是自动歌曲制作中最重要的任务之一,近年来随着神经网络模型在自动写歌谱曲任务中的进展。
近期的工作~\citep{lee-etal-2019-icomposer,Chen2020MelodyConditionedLG,songmass,telemelody,ai_lyricist,xue-etal-2021-deeprapper}绝大部分都使用了神经网络进行序列生成的模型框架,但是各自的侧重点不太一样。
有些工作关注如何限制生成结果的和节奏的对齐,也有写工作专注于限制了生成文本的主题或适配歌曲的类别。
一些工作~\citep{songmass,telemelody},如图\ref{fig:attn_diag},提出利用既定歌词和旋律间的注意力机制,
\begin{figure}[ht]
  \includegraphics[width=0.99\textwidth]{figure/related/digattn.pdf}
  \caption{歌词和旋律之间的注意力权重矩阵示意图。}
  \label{fig:attn_diag}
\end{figure}
\begin{figure}[ht]
  \includegraphics[width=0.99\textwidth]{figure/related/GuidedAttention.png}
  \caption{真实的对齐情况可以对歌词和旋律之间的注意力权重矩阵进行监督。}
  \label{fig:attn_loss}
\end{figure}
通过在注意力权重值的矩阵上进行动态规划求得最短路来找到歌词旋律之间合适的对齐方式。注意力矩阵本身除了相应任务的监督以外,也会受到真实对齐方式的监督。
然而,由于这种方法得出的对齐路径来自于某种对齐距离矩阵,在未加限制的情况下有时会导致一音符对齐多字的非单调性输出,并且自注意力机制模块在训练中也需要相对大量的数据。但最重要的一点可能是,这种方法的对齐组件是在得到翻译结果后提供类似基于规则的固定约束,而不是在训练期间和翻译一起动态地学习对齐的继承,即类似于后处理网络,而不是动态学习对齐从而限制歌词生成的模块。

\subsubsection{自适应计算时间}
本文在后文中使用的自适应分组的做法实际上是自适应计算时间算法的变种。自适应计算时间方法是\citet{act}中提出的,用于控制循环神经网络模型中每一个时刻重复运算的次数的算法,即使用算法来控制循环神经网络在时序顺序上的每一个时间步状态时的计算网络的深度。
\begin{figure}[ht]
  \includegraphics[width=0.99\textwidth]{figure/related/act.png}
  \caption{自适应计算时间控制循环神经网络的示意图。}
  \label{fig:act_rnn}
\end{figure}


本文提出的模型框架则是利用了歌词旋律对齐排列的单调性,进而设计了一个用于与翻译过程并行的进行对齐预测的轻量的神经网络。
\section{语音合成声学模型和扩散模型相关研究介绍}
\subsection{语音合成和歌声合成声学模型相关研究介绍}
\label{sec:svs_intro}
歌声合成研究起步于使用连接式的方法~\citep{macon1997concatenation,kenmochi2007vocaloid}或基于隐马尔可夫过程的参数化~\citep{saino2006hmm,oura2010recent}方法来生成声音。这两类方法在现在看来,流程都相对繁琐,且在内容上缺乏灵活性,音频听起来也不够和谐。由于深度学习的快速发展,在过去几年中,已经有多种基于深度神经网络的歌声合成系统被提出。\citet{nishimura2016singing,blaauw2017neural,kim2018korean,nakamura2019singing,gu2020bytesing}等工作率先尝试使用利用神经网络将上下文特征映射为声学参数特征。
基于深度神经网络的合成方法大致可以分成自回归和非自回归两类。
\subsubsection{时序自回归式的声学模型}
自回归类模型被提出的较早,其利用序列建模的方法,使用LSTM等循环神经网络对声音信号的线性频谱或梅尔频谱进行时序建模,一帧一帧地预测频谱的高低频情况。这一类模型的代表作就是Tacotron系列。
\begin{figure}[htbp]
  \includegraphics[width=0.99\textwidth]{figure/related/tacotron2.pdf}
  \caption{Tacotron声学模型示意图}
\end{figure}
Tacotron\citep{tacotron}构建的基于注意力机制的编码器-解码器框架将字符作为输入并输出线性频谱,使用Griffin-Lim算法\citep{GriffinLim}生成波形。Tacotron 2\citep{shen2018natural}则开始生成后续工作普遍采用的梅尔频谱并使用WaveNet\citep{vanwavenet}模型作为声码器并将梅尔频谱转换为波形。Tacotron 2和早期的连接式、参数式方法相比,合成的语音质量已经有了很大的提高。
后来也有许多工作尝试从不同方面改进和发展Tacotron系列。比如\citet{gsttacotron}、\citet{reftacotron}等在Tacotron原有框架基础上引入音频参考编码器和样式词来增强语音合成的表达能力。
\citet{nonattentivetacotron}和\citet{durian}则尝试替换Tacotron中的注意力机制,而使用一个单独的持续时间预测器进行自回归预测。
也有基于Tacotron构建端到端的文本直接生成波形的模型,如Wave Tacotron\citep{wavetacotron}。
基于Transformer结构的模型的提出最直接的动机就是使用循环神经网络构建的自回归模型有无法同时平行地训练和推理带来的模型效率问题和长频谱下自回归模型建模能力衰减的问题。
\begin{figure}[htbp]
  \includegraphics[width=0.99\textwidth]{figure/related/fs2.pdf}
  \caption{基于Transformer的声学模型示意图}
\end{figure}
\citet{transformertts}提出使用基于Transformer的编码器-解码器的基本模型结构来从输入的音素直接生成梅尔频谱。
\citet{transformertts}除了Transformer结构以外,仍沿用了Tacotron 2的一些设计,达到了和Tacotron 2的相似质量的音质,但训练和推理速度更快。然而,与基于循环神经网络Tacotron系列模型相比,Transformer中的编码器-解码器的注意力计算不够稳定和鲁棒,因此,一些工作开始致力于增强基于Transformer的声学模型的鲁棒性,如对注意力矩阵添加对角化限制\citep{robutrans}等。
但由于自回归本身的训练会带来暴露偏差(exposed bias)问题,这种建模方法有一定的缺陷,在实际应用中反映出的就是拖音、漏音、静音过长、韵律预测不稳等现象。
\subsubsection{非自回归式的声学模型}
非自回归类模型的提出相对较晚,但相关工作近来层出不穷,从基于Transformer结构的模型到基于各类生成模型的结构,有后来居上的趋势。
无论是Tacotron系列,还是基于Transformer的自回归模型,都存在上节提到的两个问题:推理速度慢和鲁棒性问题。
自回归的梅尔频谱生成速度较慢,特别是对于较长语音序列有较多的语音帧需要生成时。生成的语音存在一定的漏音、重复等造成音频听起来很不自然的问题,这主要是由基于Transformer的编码器-解码器的自回归生成中,文本和梅尔频谱之间的注意力对齐不准确造成监督不准确引起的。
因此\citet{ren2019fastspeech}
\begin{figure}[htbp]
  \includegraphics[width=0.99\textwidth]{figure/related/bvaetts.png}
  \caption{基于变分自编码器的声学模型示意图}
\end{figure}

\begin{figure}[htbp]
  \subfloat{
    \includegraphics[width=0.49\textwidth]{figure/related/glowtts_a.pdf}
  }
  \subfloat{
    \includegraphics[width=0.49\textwidth]{figure/related/glowtts_b.pdf}
  }
  \caption{基于生成式流模型搭建的声学模型示意图}
\end{figure}

% \begin{figure}[htbp]
%   \includegraphics[width=0.99\textwidth]{figure/related/gantts.png}
%   \caption{基于生成对抗训练的声学模型示意图}
% \end{figure}
\subsection{歌声合成模型}
\citet{ren2020deepsinger}成功地使用了从音乐网站挖掘的歌唱数据从零开始构建了歌声合成系统,也为后续工作提供了一个方便的流程框架。\citet{blaauw2020sequence}则提出了一种基于前馈Transformer的非自回归歌声合成模型,推断过程快速,且能避免自回归模型引起的暴露偏差问题。
此外,对抗训练也是生成模型中的常用技术,在对抗训练的帮助下,\citet{lee2019adversarially}~提出了一个直接生成线性频谱的端到端框架。\citet{wu2020adversarially}~则提出了一个可以利用数量有限的录音数据就能构建起来的支持多歌手的歌声合成系统,并通过使用多随机窗口鉴别器来提高合成语音质量。\citet{chen2020hifisinger}~在模型中引入了多尺度对抗训练,以相对较高的采样率(48kHz)合成高清音频。
% \begin{figure}[htbp]
%   \includegraphics[width=0.99\textwidth]{figure/related/gantts.pdf}
%   \caption{基于对抗训练的歌声合成模型示意图}
% \end{figure}
近年来,歌声合成系统的音频结果的语音自然度和多样性都在不断提高,已显现在24kHz和48kHz下媲美人声的潜力。

\subsection{扩散模型相关研究介绍}
扩散概率模型也一种是通过优化似然函数整体变分下界进行训练的参数化的马尔可夫链,这样一个参数化的随机过程能以恒定的时间步长生成与数据分布匹配的样本~\citep{Ho2020ddpm}。
扩散模型首先由\citet{sohl2015deep}一文在2015年提出,之后的工作\citet{Ho2020ddpm}~又对最初的扩散模型进行了改进,以使用特定的参数化方式生成高质量图像,也揭示了扩散模型与去噪梯度得分匹配之间的等价性~\citep{song2019generative,song2021scorebased}。
\begin{figure}[htbp]
  \includegraphics[width=0.99\textwidth]{figure/related/ddpm.pdf}
  \caption{基于扩散过程的图像去噪合成示意图}
\end{figure}
近来,随着扩散模型本身理论不断成熟,越来越多的工作开始尝试在应用任务中使用扩散模型来利用其高质量生成的优点并解决其运行速度慢的问题。\citet{kong2021diffwave}~和\citet{chen2021wavegrad}~两篇工作尝试将扩散模型应用于神经声码器这样的应用任务,根据梅尔频谱生成高保真的声音波形。
\begin{figure}[htbp]
  \includegraphics[width=0.99\textwidth]{figure/related/diffwave.png}
  \caption{基于扩散过程的声音波形合成示意图}
\end{figure}
\citet{chen2021wavegrad}~还提出了一种连续的噪声规划来减少推理迭代所需要的时间步数,同时保持原有的合成质量。\citet{song2021denoising}~通过提供更快的采样机制和有意义地在样本之间插值的方法来扩展扩散模型。扩散模型是一种新兴的技术,已应用于无条件图像生成、条件指导下的梅尔频谱到波形生成(神经声码器)等领域。在本文的工作中,\ref{sec:svs}节为提出了一个基于扩散模型声学模型,该模型在给定乐谱和文本(或仅给定文本)的情况下生成梅尔频谱。
\section{本章小结}
本章主要分多个领域分别介绍歌曲歌词生成及限制性翻译和歌声合成相关技术的背景、基本原理、国内外研究现状和近期进展。

\chapter{自动歌曲翻译}
\section{歌曲翻译数据集}
\section{歌曲翻译数据的收集和预处理}
\section{Transformer Encoder-Decoder模型的结构}
\section{Adaptive Computation Time算法原理}
\section{歌曲翻译的评价指标}
\section{本章小结}

\chapter{歌声合成}
\section{歌声合成数据集}
\section{歌声合成数据的收集和预处理}
\section{扩散模型}
\section{歌声合成的评价指标}
质量MOS,速度RTF
\section{本章小结}

% \chapter{歌曲到歌声翻译系统}
\section{歌曲翻译结果后处理}

\chapter{总结和展望}
\section{总结}
机器翻译一直以来都是自然语言处理领域针对不同语言文字的一项重要技术。自动歌曲翻译是神经机器翻译在这一基础上针对非常规语体的拓展性研究,这一任务旨在快速地、自动地、高质量地将歌曲的歌词文本翻译到另外一种语言中,同时要求翻译后歌词文本搭配相应旋律依然能以歌曲的形式呈现出来,即翻译后歌词仍能以某种方式来进行艺术性的演唱并传达原意。歌声合成使得直观地呈现歌曲翻译结果成为可能。同时,由于歌声合成可以承接歌曲翻译的输出而连接成为完整的级联式歌曲到歌声的翻译系统,这为歌曲翻译的实际应用打下了坚实的基础。自动歌曲翻译搭配良好的歌声合成效果进行翻唱自动合成,那么听众无需学习多国语言、乐理知识或具备演唱能力就能欣赏来自不同语言不同文化的经典歌曲的母语翻唱版本。

随着自然语言和翻译研究进入深度学习时代,自动化歌曲翻译研究却相对发展缓慢,目前仅有极少量工作对这一方向进行过探讨。此外,翻译领域如WMT14、WMT16等多语言翻译公开数据集层出不穷,但涉及词曲类的公开翻译语料则非常寥寥。歌曲本身的电子化需要有一定专业知识的人员完成,即使借助于神经网络模型,在实际错误率较高的情况下也需要人员校对,因此研究本身就具有一定的门槛。在充分感受、理解歌曲的基础上进行的翻译则更有难度。

本文的主要工作总结如下:

(1)

(2)

本文提出的歌词和歌词-旋律共同翻译模型和基于扩散模型的翻唱歌声合成声学模型在实验中的主客观评价标准下都取得了较好的实验效果。\ref{sec:ast}章中。\ref{sec:svs}章中。同时,两个模型可以进行衔接以形成级联式的自动歌曲翻译翻唱系统,有很强的工业应用价值。
\section{挑战与未来展望}
本文针对自动歌曲翻译和歌声合成这两项任务分别提出了新的模型方法,尽管两个模型在实验中相比于其他现有算法都表现出了很有竞争力的效果,但是仍存在一些不足之处,有进一步优化的空间。主要有以下几点:

本文提出的自动歌曲翻译模型的结果评价在翻译和歌词-旋律对齐质量上都比较依赖人工评测。贵,评测量小
本文在自动歌曲翻译任务中提出的数据收集流程对于收集成千上万句平行语料来说比较合适,对于收集更大量级的数据来说仍比其他这样量级的数据收集流程繁琐很多,且所需技能也需要较长时间的专业训练,更好的自动歌曲翻译数据标注流程亟待探索优化。

本文提出的基于扩散模型的翻唱合成方法虽然合成效果好,训练稳定,但是由于扩散模型迭代进行的去噪过程,为了保证合成效果,去噪模型需要迭代运行数百次,其推理速度和一般的语音合成模型相比有较大的劣势,对于实际落地应用是一个很大的效率问题。之后的研究可以聚焦于扩散模型本身的运行机制或针对这样的合成场景探索在保持合成质量的情况下缩短扩散和逆向过程,或构建更加轻量级、运行更快的去噪模块。

本文提出的两个模型搭建出的级联式系统本身存在误差累积的问题。在自动语音识别、歌曲翻译和歌声合成分别已有相应研究的情况下,探索端到端的歌曲到歌曲翻译模型已成为可能。

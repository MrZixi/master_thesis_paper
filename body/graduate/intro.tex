\chapter{绪论}
\section{引言}
文字和语音都是自然语言处理领域的研究对象。无论是比较早期的句法、词法分析,音素研究,还是诞生稍晚的针对语音内容的识别、拼接合成,对于文字和语音内容的研究可称得上互为表里。语音是文字可感知的信号载体,文字是语音内容的本质表示。
机器翻译一直以来都是自然语言处理领域针对不同语言文字的一项重要技术。对于机器翻译的研究最早始于20世纪30年代,它的发展一直与计算机技术、语言学和信息论等学科密切相关。从早期的字典匹配、结合专家知识的规则翻译等朴素方法,到结合概率统计学和语言学的统计机器翻译,再到近年来随着运算硬件性能提高和深度神经网络发展而兴起的神经机器翻译,机器翻译技术取得了世人瞩目的成功,这项技术本身也逐渐从学界中的理论发展走向工业界的落地实践。

自动歌曲翻译是神经机器翻译在这一基础上针对非常规语体的拓展性研究,这一任务旨在。

歌声合成则是由语音合成任务衍生而来。语音合成是仅以文本为输入、以梅尔特征图或声波波形为输出的生成任务,歌声合成和语音合成任务不同的是,其文本发声时所对应的音高和时长都为设定好的乐谱所限制。
歌声合成使得歌曲翻译结果的直观评测成为可能。同时,由于歌声合成可以承接歌曲翻译的输出结果而连接成为完整的、级联式的歌曲到歌声的翻译系统,这为歌曲翻译的实际应用打下了坚实的基础。
\section{国内外研究现状}
\section{研究意义及内容}
\section{章节安排}
本文各章节组织如下:


第一章:绪论。第一章节主要介绍了歌曲翻译技术和歌声合成技术的定义、应用和发展情况、自动歌曲翻译和歌声合成的国内外研究现状、以及自动歌曲翻译和歌声合成的研究背景及意义。另外,第一章节还阐述了本文将基于xxx来研究自动歌曲翻译任务、基于扩散模型来研究歌声合成任务。

第二章:

第三章:

第四章:
\section{章节小结}
